\section{Анализ предметной области}
\subsection{Актуальность цифровизации медицины в наше время}

Актуальность темы. На данный момент в нашей стране не существует программного обеспечения для врачей-диетологов. Есть программа для записи пациента на прием, отдельный сайт со стандартными диетами, которые назначают по установленному заболеванию желудочно-кишечного тракта у клиента, так же отдельно можно найти калькулятор калорий. Одного ПО по всем этим пунктам нет, поэтому было принято решение, помочь центрам здорового питания в упрощении их работы, создав веб-сайт, где совмещены все необходимые функции для работы врачей-диетолог с пациентами.

Цифровизация медицины – это процесс внедрения и применения ИТ-технологий, цифровых сервисов в отрасли, которая затрагивает все процессы – от управления системой здравоохранения до практической деятельности врачей на местах. Цифровизация предполагает качественную трансформацию медицины, повышение ее эффективности за счет оптимизации и автоматизации системы, организации четкой работы всех ее звеньев как в государственном, так и частном сегменте.

В наши дни система здравоохранения сталкивается с ежедневными вызовами, требующими от нее проактивных решений. Цифровизация медицины в России, начавшаяся десятилетия назад и на сегодняшний день набравшая высокие обороты, – логичный ответ на воздействующие внешние факторы. Можно сказать, что процесс оптимизации, качественной трансформации отрасли набрал динамику и стал одним из центральных буквально за последние несколько лет. Пережитые годы пандемии и в разы возросшая нагрузка на систему оказания медицинской помощи населению, потребовавшая оперативного внедрения современных ИТ-решений в лечебные процессы, всеобщая тенденция на цифровизацию жизни, использование гаджетов, в том числе в вопросах заботы о здоровье, разработка и распространение многочисленных онлайн приложений и сервисов, телемедицинские консультации, развитие систем искусственного интеллекта – все это происходит здесь и сейчас, на наших глазах внедряются инновационные технологии в медицине, что оказывает колоссальное влияние на качественную трансформацию отрасли и как следствие на совершенствование лечебного процесса, качество обслуживания пациентов, а также управление системой в целом.


\subsection{Цифровизация в медицине и здравоохранении: задачи и вызовы}

В конце 2021 года Правительство Российской Федерации определило стратегию развития системы (цифровизация в медицине и здравоохранении) на ближайшие несколько лет, выделив основные направления в области цифровой трансформации. Во-первых, это создание в здравоохранении единого цифрового контура на основе ЕГИСЗ, а также разработка и внедрение на федеральном уровне медицинских платформенных решений.

Благодаря реализации данных стратегических инициатив будет решен ряд стоящих перед отраслью задач:

\begin{enumerate}
	\item Обеспечение взаимодействия различных медицинских учреждений между собой и государственными системами.
	\item Станет возможным дистанционный лицензионный контроль медицинской деятельности.
	\item Обеспечение внедрения системы электронных рецептов на всей территории страны.
	\item Внедрение системы контроля полноты выполнения клинических рекомендаций на всех этапах лечебного процесса.
\end{enumerate}

Кроме того, современные технологии в медицине позволят оптимизировать процессы управления системой здравоохранения в целом, учитывать первичные медицинские данные, активно использовать структурированные электронные медицинские документы, а также осуществить постепенный переход на электронный документооборот в отрасли.

Цифровизация медицины, по мнению экспертов, должна способствовать следующим аспектам:

\begin{enumerate}
	\item Существенно минимизировать затраты на работу системы.
	\item Значительно повысить качество оказываемых медицинских услуг.
	\item Обеспечить более широкую доступность медицинской помощи.
	\item Оптимизировать время, которое тратит пациент на получение услуг (к примеру, ускорение процесса через внедрение сервиса записи на прием к специалисту).
	\item Сократить время работы самого врача (внедрение систем поддержки принятий решений, ведение электронной карты с ее быстрым заполнением через систему преобразования голоса в текст).
\end{enumerate}

Процесс цифровизации системы здравоохранения в нашей стране стартовал еще в 2011 году, именно в это время было сформулировано понятие «Цифровой контур здравоохранения». Конечно, не все намеченные задачи до сих пор удалось превратить в жизнь. Однако, отрицать существенный прогресс в этом процессе также бессмысленно. Получившие повсеместное распространение электронные больничные листы, внедрение электронных справок, свидетельств о рождении и смерти. Мгновенная отправка и обмен оцифрованными рентгеновскими снимками, которые позволили ставить рентгенологу точный диагноз удаленно, что обеспечило возможность создания в период эпидемии коронавируса специализированных центров, диагностирующих пациентов с патологиями на всей территории РФ, развитие системы искусственного интеллекта, системы маршрутизации пациентов, создание порталов для врачей и пациентов, содержащие полные и достоверные сведения о здоровье, пройденных процедурах, анализах, диагнозах. Все это, безусловно, уже сегодня качественно меняет существующую систему.
Кроме того, не так давно Министерство экономического развития инициировало эксперимент по дистанционной торговле рецептурными лекарствами, опубликовав законопроект с соответствующими поправками к существующему законодательству. Так в рамках экспериментального правового режима можно будет дистанционно получать и отоваривать рецепты на такие препараты, как антибиотики, лекарства для лечения хронических заболеваний сердца, бронхиальной астмы, диабета, и другие.

\subsection{Обоснование выбора технологии проектирования}

На сегодняшний день информационный рынок, поставляющий программные решения в выбранной сфере, предлагает множество продуктов, позволяющих достигнуть поставленной цели – разработки web-сайта.

\subsubsection{Описание используемых технологий и языков программирования}

В процессе разработки web-сайта используются программные средства и языки программирования TypeScript, SQL. Каждое программное средство и каждый язык программирования применяется для круга задач, при решении которых они необходимы.

\subsubsection{JavaScript-библиотека -- React}

React — это JavaScript-библиотека для создания пользовательских интерфейсов. Обратите внимание, что это именно библиотека, а не фреймворк. React часто называют фреймворком, но это ошибка. Во-первых, его использование ни к чему вас не обязывает, не формирует «фрейм» проекта. Во-вторых, React выполняет единственную задачу: показывает на странице компонент интерфейса, синхронизируя его с данными приложения, и только этой библиотеки в общем случае недостаточно для того, чтобы полностью реализовать проект.

Вскоре после появления React и подобные ему решения (Vue.js, Svelte) практически захватили мир фронтенда: потому что они помогают решать проблемы, основываясь на идее декларативного программирования, а не на императивном подходе.

— Декларативный подход состоит в описании конечного результата (что мы хотим получить).

— При императивном подходе описываются конкретные шаги для достижения конечного результата (как мы хотим что-то получить).

Оказалось, что декларативный подход отлично подходит для создания интерфейсов, и он прижился в сообществе. Этот подход работает не только в вебе: сравнительно недавно компания Apple представила фреймворк SwiftUI, основанный на тех же принципах.


\subsubsection{Язык программирования JavaScript}

\paragraph{Достоинства языка JavaScript}

JavaScript – объектно-ориентированный язык программирования для написания сценариев \cite{javascript}. Чаще всего JavaScript используется для написания сценариев работы с web-страницами, отображаемыми web-браузером. Web-бра\-у\-зер интерпретирует код сценария языка JavaScript, и на основе описанных в сценарии действий производит манипуляции с разметкой web-страницы. Посредством языка JavaScript реализуется возможность программирования на стороне клиента. Предоставляет возможность доступа к элементам разметки web-страницы посредством объектов. При создании сценариев на языке JavaScript приходится сталкиваться с трудностями, связанными с тем, что различные web-браузеры могут по-разному интерпретировать эти сценарии. Серьезные трудности возникают, если какой-либо из браузеров не поддерживает тот или иной объект, метод или свойство. Наиболее практичным способом решения данной проблемы является использование библиотеки jQuery. Данная библиотека реализована на языке JavaScript и расширяет возможности данного языка, нивелируя различия между браузерами.

\paragraph{Недостатки языка Javascript}
\begin{enumerate}
	\item Язык компилируется в момент исполнения кода. Каждый раз, когда вы открываете сайт, javascript код начинает компилироваться. Как минимум увеличивается время выполнения программы.
	\item Отсутствует типизация данных. Проблема всех скриптовых языков. Пока выполнение кода не дойдет до нужной строчки, не узнаешь работает ли она. А ведь значительную часть по поиску ошибок мог бы взять на себя компилятор, если бы знал типы данных, с которыми он работает. Да и по скорости выполнения, типизированный код быстрее.
	\item Не привычная для многих программистов объектная модель. Классы и наследование классов присутствует, но оно сильно отличается от привычной многим реализаций в языках программирования C++ Java.
\end{enumerate}