\addcontentsline{toc}{section}{СПИСОК ИСПОЛЬЗОВАННЫХ ИСТОЧНИКОВ}

\begin{thebibliography}{9}

    \bibitem{javascript} Фримен, А. Практикум по программированию на JavaScript / А. Фримен. – Москва~: Вильямс, 2013. – 960 с. – ISBN 978-5-8459-1799-7. – Текст~: непосредственный.
    \bibitem{bigbook} Руденков, Н. А. Основы сетевых технологий : учебник / Н. А. Руденков, Л. И. Долинер. – Екатеринбург : УрФУ, 2011. – 300 с. ISBN 5-24526- 456-1. – Текст : непосредственный.
    \bibitem{css} Веру, Л. Секреты CSS. Идеальные решения ежедневных задач / Л. Веру. – Санкт-Петербург : Питер, 2016. – 336 с. – ISBN 978-5-496-02082-4. – Текст~: непосредственный.
    \bibitem{mysql}	Гизберт, Д.  и SQLite / Д. Гизберт. – Москва~: НТ Пресс, 2013. – 320 с. – ISBN 978-5-477-01174-2. – Текст~: непосредственный.
	\bibitem{html5}	Голдстайн, А. HTML5 и CSS3 для всех / А. Голдстайн, Л. Лазарис, Э. Уэйл. – Москва~: Вильямс, 2012. – 368 с. – ISBN 978-5-699-57580-0. – Текст~: непосредственный.
	\bibitem{htmlcss}	Дэкетт, Д. HTML и CSS. Разработка и создание веб-сайтов / Д. Дэкетт. – Москва~: Эксмо, 2014. – 480 с. – ISBN 978-5-699-64193-2. – Текст~: непосредственный.
	\bibitem{bigbook}	Макфарланд, Д. Большая книга CSS / Д. Макфарланд. – Санкт-Петербург : Питер, 2012. – 560 с. – ISBN 978-5-496-02080-0. – Текст~: непосредственный.
	\bibitem{uchiru}	Лоусон, Б. Изучаем HTML5. Библиотека специалиста / Б. Лоусон, Р. Шарп. – Санкт-Петербург : Питер, 2013 – 286 с. – ISBN 978-5-459-01156-2. – Текст~: непосредственный.
	\bibitem{chaynik}	ibooks.ru : электронно-библиотечная система : сайт. – СанктПетербург, 2010 – . – URL: ibooks.ru (дата обращения: 23.02.2021). – Текст: электронный.    
	\bibitem{22}	Титтел, Э. HTML5 и CSS3 для чайников / Э. Титтел, К. Минник. – Москва~: Вильямс, 2016 – 400 с. – ISBN 978-1-118-65720-1. – Текст~: непосредственный.    
	\bibitem{1231}	Консультант студента : электронно-библиотечная система : сайт. – Москва, 2013 – . – URL: https://www.studentlibrary.ru (дата обращения: 23.02.2021). – Текст: электронный.  
	\bibitem{sdf}	Руденков, Н. А. Основы сетевых технологий : учебник / Н. А. Руденков, Л. И. Долинер. – Екатеринбург : УрФУ, 2011. – 300 с. ISBN 5-24526-
	456-1. – Текст : непосредственный.   
	\bibitem{servsssds}	Фаулер, М. UML. Основы / М. Фаулер ; пер. с англ. А. Петухова. – 3-е изд. – Санкт-Петербург : Символ-Плюс, 2004. – 192 с. ISBN 5-93286-060- Х. – Текст : непосредственный.
	\bibitem{sdf}	Иванова, Г. С. Проектирование программного обеспечения : учебное пособие / Г. С. Иванова, Т. Н. Ничушкина. – Москва : МГТУ им. Н
	\bibitem{sdf}	Баумана, 2013. – 102,[1] с. – ISBN 5-7038-2285-8. – Текст : непосредственный.
	\bibitem{sdf}	Сырых, Ю.А. Современный веб-дизайн. Эпоха Веб 3.0 / Ю.А. Сырых. – М.: Диалектика, 2014. – 368 c. ISBN 978-5-8459-1809-3. – Текст : непосредственный.
	\bibitem{sdf}	Эспозито, Д. Разработка современных веб-приложений: анализпредметных областей и технологий / Д. Эспозито. – М.: Вильямс И.Д.,2017. – 464 c. ISBN 978-5-9908910-3-6. – Текст : непосредственный.
	\bibitem{sdf}	Фельке-Моррис, Т. Большая книга веб-дизайна / Т.ФелькеМоррис. – М.: Эксмо, 2014. – 512 c ISBN 978-5-699-55404-1. – Текст : непосредственный
	\bibitem{sdf}	Сырых, Ю.А. Современный веб-дизайн. Настольный и мобильный / Ю.А. Сырых. – М.: Вильямс, 2014. – 384 c. ISBN 978-5-8459-1905-2. – Текст : непосредственный.
	\bibitem{sdf}	Седерхольм, Д. Пуленепробиваемый веб-дизайн / Д.Седерхольм. – СПб.: Питер, 2012. – 304 c. ISBN 978-5-459-01271-2. – Текст : непосредственный.
	\bibitem{sdf}	Мацяшек, А. Анализ и проектирование информационных систем с помощью UML 2.0 / А. Мацяшек. – М.:Вильямс, 2008. – 816 с. ISBN 978-5-	8459-1430-9. – Текст : непосредственный.
	\bibitem{sdf}	Криспин, Л. Гибкое тестирование: практическое руководство для тестировщиков ПО и гибких команд/ Л. Криспин. – М.: Вильямс, 2010. – 572 с. ISBN 978-5-8459-1625-9. – Текст : непосредственный.
	\bibitem{sdf}	Znanium.com : электронно-библиотечная система : сайт. – Москва, 2011 – . – URL: https://znanium.com (дата обращения: 23.02.2021). – Текст: электронный.
	\bibitem{sdf}	METANIT.COM : Сайт о программировании : сайт. – Москва, 2011 – . – URL: https://metanit.com (дата обращения: 23.02.2021). – Текст: электронный.
	\bibitem{sdf}Берд, Дж. Веб-дизайн. Руководство разработчика. / Дж.Берд. – СПб.: Питер, 2012. – 224 c. ISBN 978-5-459-00901-9. – Текст : непосредственный.
	\bibitem{sdf}	Дакетт, Д. HTML и CSS. Разработка и дизайн веб-сайтов /Д. Дакетт. – М.: Эксмо, 2015.– 480 c. ISBN 978-1-118-00818-8. – Текст : непосредственный.
	\bibitem{sdf}	Кирсанов, Д. Веб-дизайн: книга Дмитрия Кирсанова / Д.Кирсанов. – М.: Символ, 2015. – 368 c. ISBN 5-93286-003-0. – Текст : непосредственный.
	\bibitem{sdf}	Макнейл, П. Настольная книга веб-дизайнера / П.Макнейл. – СПб.: Питер, 2013. – 264 c. ISBN 978-5-4461-0149-8. – Текст : непосредственный.
	\bibitem{sdf}	Нильсен, Я. Веб-дизайн: книга Якоба Нильсена / Я.Нильсен. – М.: Символ, 2015. – 512 c. ISBN 978-5-8459-1222-0. – Текст : непосредственный.
	\bibitem{sdf}	Миковски, М.С. Разработка одностраничных веб-приложений / М.С. Миковски, Д.К. Пауэлл. – М.: ДМК, 2014. – 512 c. ISBN 978-5-97060- 072-6. – Текст : непосредственный.
\end{thebibliography}
