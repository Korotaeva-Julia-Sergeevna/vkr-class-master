\section{Технический проект}
\subsection{Концептуальная модель предметной области}

В таблице \ref{ssevsws:tablekm} представлены основные сущности и типы связей для них, используемые в информационной части (базе данных) программной системы.

\begin{xltabular}{\textwidth}{|c|X|X|}
	\caption{Основные сущности и типы ассоциаций для них, используемые в информационной части (базе данных) программной системы\label{ssevsws:tablekm}}\\ \hline
	\centrow Основные сущности  & \centrow  Связь (ассоциация) & \centrow Кратность связи \\ \hline
	\endfirsthead
	\continuecaption{Продолжение таблицы \ref{ssevsws:tablekm}}
	\centrow Основные сущности  & \centrow  Связь (ассоциация) & \centrow Кратность связи \\ \hline
	\finishhead
	Users -- Patient  &  использует &  1 -- 0..* \\ \hline 
	Users -- Diets   & использует &  1 -- 0..* \\ \hline 
	Patient – Diseases & использует &  1..* -- 1..* \\ \hline 
	Diets -- Products  & использует &  1..* -- 1..* \\ \hline 
	Diets -- Diseases  & использует &  1 -- 0..*
\end{xltabular}

На рисунке \ref{fig:model} представлена концептуальная модель предметной области информационной части (базы данных) программной системы.

\begin{figure}[H]
	\centering
	\includegraphics[width=0.7\linewidth]{"images/Концептуальная модель предметной области.drawio"}
	\caption{Концептуальная модель предметной области информационной части (базы данных) программной системы}
	\label{fig:model}
\end{figure}

Ограничения в виде функциональных зависимостей: 

Не может быть двух и более пользователей (user) с одинаковыми идентификаторами (userId).

userId --> password

Не может быть двух и более пользователей (user) с одинаковыми логинами (login).

login --> password

Не может быть двух и более продуктов (product) с одинаковыми идентификаторами (productId).

productId --> id

Не может быть двух и более болезней (disease) с одинаковыми идентификаторами (diseaseId).

diseaseId --> id

Не может быть двух и более пациентов (patient) с одинаковыми идентификаторами (patientId).

patientId --> id

Не может быть двух и более диет (diet) с одинаковыми идентификаторами (dietId).

dietId --> id

На рисунке \ref{fig:imagefn} приведено формальное описание функциональных зависимостей, которым удовлетворяет база данных.

\begin{figure}[H]
	\centering
	\includegraphics[width=0.7\linewidth]{"images/Формальное описание функциональных зависимостей"}
	\caption{Формальное описание функциональных зависимостей}
	\label{fig:imagefn}
\end{figure}

Ограничения ссылочной целостности:

При удалении пользователя из базы данных должны быть удалены все отношения, относящиеся к этому пользователю. Необходимо использовать ограничение CASCADE. При удалении пользователя из базы данных должны быть удалены все связи: patientId, dietId, относящиеся к этому пользователю. При удалении связи из базы данных должны быть удалены все отношения, относящиеся к этой связи. Необходимо использовать ограничение CASCADE

На рисунке \ref{fig:diagrammodel} представлена концептуальная модель работы разрабатываемой информационно-вычислительной системы в виде диаграммы состояний.

\begin{figure}[H]
	\centering
	\includegraphics[width=0.7\linewidth]{"images/Концептуальная модель работы разрабатываемой информационно-вычислительной системы в виде диаграммы состояний.drawio"}
	\caption{Концептуальная модель работы разрабатываемой информационно-вычислительной системы в виде диаграммы состояний}
	\label{fig:diagrammodel}
\end{figure}

\subsection{Содержание информационных блоков. Основные сущности}

Проанализировав требования, можно выделить шесть основных сущностей:
\begin{itemize}
	\item "<Пользователь">;
	\item "<Продукты">;
	\item "<Болезни">
	\item "<Пациенты">
	\item "<Диеты">.
\end{itemize}

В состав сущности "<Пользователь"> можно включить атрибуты, представленные в таблице \ref{news:table}.

\begin{xltabular}{\textwidth}{|l|l|p{1.7cm}|X|}
	\caption{Атрибуты сущности "<Пользователь">\label{news:table}}\\ \hline
	\centrow Поле & \centrow Тип & \centrow Обяза\-тельное & \centrow Описание \\ \hline
	\thead{1} & \thead{2} & \centrow 3 & \centrow 4 \\ \hline
	\endfirsthead
	\continuecaption{Продолжение таблицы \ref{news:table}}
	\thead{1} & \thead{2} & \centrow 3 & \centrow 4 \\ \hline
	\finishhead
	\_id & ObjectId & true & Уникальный идентификатор \\ \hline 
	login & String & true & login пользователя\\ \hline 
	password & String & true & Пароль пользователя \\ \hline 
	fio & String & true & ФИО пользователя \\ \hline 
	isAdmin & Tynyint & true & Указатель Является ли данный пользователь администратором \\ \hline 
\end{xltabular}

В состав сущности "<Продукты"> можно включить атрибуты, представленные в таблице \ref{news:tablepr}.

\begin{xltabular}{\textwidth}{|l|l|p{1.7cm}|X|}
	\caption{Атрибуты сущности "<Продукты">\label{news:tablepr}}\\ \hline
	\centrow Поле & \centrow Тип & \centrow Обяза\-тельное & \centrow Описание \\ \hline
	\thead{1} & \thead{2} & \centrow 3 & \centrow 4 \\ \hline
	\endfirsthead
	\continuecaption{Продолжение таблицы \ref{news:tablepr}}
	\thead{1} & \thead{2} & \centrow 3 & \centrow 4 \\ \hline
	\finishhead
	\_id & ObjectId & true & Уникальный идентификатор \\ \hline 
	name & String & true & Наименование продукта\\ \hline 
	calories & Float & true & Калории \\ \hline 
	proteins & float & true & Белок\\ \hline 
	fats & float & true & Жиры \\ \hline 
	carbohydrates & float & true & Углеводы \\ \hline 
\end{xltabular}

В состав сущности "<Болезни"> можно включить атрибуты, представленные в таблице \ref{news:tableb}.

\begin{xltabular}{\textwidth}{|l|l|p{1.7cm}|X|}
	\caption{Атрибуты сущности "<Болезни">\label{news:tableb}}\\ \hline
	\centrow Поле & \centrow Тип & \centrow Обяза\-тельное & \centrow Описание \\ \hline
	\thead{1} & \thead{2} & \centrow 3 & \centrow 4 \\ \hline
	\endfirsthead
	\continuecaption{Продолжение таблицы \ref{news:tableb}}
	\thead{1} & \thead{2} & \centrow 3 & \centrow 4 \\ \hline
	\finishhead
	\_id & ObjectId & true & Уникальный идентификатор \\ \hline 
	name & String & true & Наименование болезни\\ \hline 
	dietID & Number & true & Указывает на диету, которая показа при данном заболевании \\ \hline 
\end{xltabular}

В состав сущности "<Пациенты"> можно включить атрибуты, представленные в таблице \ref{news:tablep}.

\begin{xltabular}{\textwidth}{|l|l|p{1.7cm}|X|}
	\caption{Атрибуты сущности "<Пациенты">\label{news:tablep}}\\ \hline
	\centrow Поле & \centrow Тип & \centrow Обяза\-тельное & \centrow Описание \\ \hline
	\thead{1} & \thead{2} & \centrow 3 & \centrow 4 \\ \hline
	\endfirsthead
	\continuecaption{Продолжение таблицы \ref{news:tablep}}
	\thead{1} & \thead{2} & \centrow 3 & \centrow 4 \\ \hline
	\finishhead
	\_id & ObjectId & true & Уникальный идентификатор \\ \hline 
	fio & String & true & ФИО пациента \\ \hline 
	age & Number & true & Возраст пациента \\ \hline 
	sex & String & true & Пол пациента\\ \hline 
	weight & Number & true & Вес пациента \\ \hline 
	activity & String & true & Активности в течении дня пациента \\ \hline 
	goal & String & true & Цели пациента \\ \hline 
	userId & Tynyint & true & Указатель к какому врачу-диетологу прикреплен пациент \\ \hline 
\end{xltabular}

В состав сущности "<Диеты"> можно включить атрибуты, представленные в таблице \ref{news:tabledi}.

\begin{xltabular}{\textwidth}{|l|l|p{1.7cm}|X|}
	\caption{Атрибуты сущности "<Диеты">\label{news:tabledi}}\\ \hline
	\centrow Поле & \centrow Тип & \centrow Обяза\-тельное & \centrow Описание \\ \hline
	\thead{1} & \thead{2} & \centrow 3 & \centrow 4 \\ \hline
	\endfirsthead
	\continuecaption{Продолжение таблицы \ref{news:tabledi}}
	\thead{1} & \thead{2} & \centrow 3 & \centrow 4 \\ \hline
	\finishhead
	\_id & ObjectId & true & Уникальный идентификатор \\ \hline 
	name & String & true & Наименование диеты\\ \hline 
	userID & Number & true & Указывает на врача-диетолога, который лечит \\ \hline 
\end{xltabular}

В системе предусмотрен внутренний механизм связи между разделами и элементами информационных блоков, поэтому введения дополнительных идентификаторов при реализации связей между сущностями не предполагается.

Экземпляры сущностей реализуются в информационных блоках посредством элементов, атрибуты сущности – посредством полей и свойств элемента. 

\subsection{Моделирование вариантов использования}

В системе должно быть представлено два вида действующих лиц:

Неавторизованный пользователь и пользователь.

На основании исследования предметной области в программе должны быть реализованы следующие прецеденты:

\begin{itemize}
	\item Прецедент «Авторизация». Данный прецедент позволяет пользователю авторизоваться в системе;
	\item Прецедент «Выход». Данный прецедент позволяет пользователю выйти из системы;
	\item Прецедент «Работа с пользователями». Данный прецедент позволяет пользователю получить список пользователей, добавлять пользователей, редактировать пользователей, удалять пользователей;
	\item Прецедент «Работа с продуктами». Данный прецедент позволяет пользователю: получить список продуктов, добавлять продукты, редактировать продукты, удалять продукты;
	\item Прецедент «Работа с болезнями». Данный прецедент позволяет пользователю: получить список болезней, добавлять болезни, редактировать болезни, удалять болезни;
	\item Прецедент «Работа с пациентами». Данный прецедент позволяет пользователю получить список пациентов, добавлять пациентов, редактировать информацию о пациентах, удалять пациентов, назначать диеты, устанавливать болезни;
	\item Прецедент «Работа с диетами». Данный прецедент позволяет пользователю получить список диет, добавлять диеты, редактировать диеты, удалять диеты.
\end{itemize}

На рисунке \ref{fig:dieagrammv} представлены диаграмма вариантов использования для программной системы

\begin{figure}[H]
	\centering
	\includegraphics[width=0.7\linewidth]{"images/Диаграмма вариантов использования.drawio"}
	\caption{Диаграмма вариантов использования}
	\label{fig:dieagrammv}
\end{figure}

Сценарии прецедентов программы информационной системы:

Сценарий для прецедента «Авторизация»

Основной исполнитель: Существующий пользователь.

Требования: Авторизованному пользователю необходимо авторизоваться в системе.

Предусловие: Запись о пользователе существует.

Постусловие: Пользователь авторизовался в системе.

Основной успешный сценарий:
\begin{enumerate}
	\item Пользователь отправил свои данные для авторизации через форму.
	\item Пользователь получил токен верификации.
	\item Пользователь получил сообщение о неправильном логине и/или пароле.\\
\end{enumerate}

Сценарий для прецедента «Выйти из аккаунта»

Основной исполнитель: Пользователь.

Требования: Пользователю выйти из своего аккаунта в системе.

Предусловие: Пользователь авторизован.

Постусловие: Пользователь вышел из своего аккаунта.

Основной успешный сценарий:
\begin{enumerate}
	\item Пользователь отправил запрос на сервер о выходе из аккаунта.
	\item Система удалила токен пользователя.\\
\end{enumerate}


Сценарий для прецедента «Работа с пользователями»

Основной исполнитель: Администратор.

Требования: Администратору необходимо добавить пользователя.

Предусловие: Администратор авторизован.

Постусловие: Администратор добавил пользователя.

Основной успешный сценарий:
\begin{enumerate}
	\item Администратор отправил запрос на сервер о добавлении нового пользователя.
	\item Система вернула обновленное содержимое списка пользователей.
	\item Администратор получил сообщение о некорректном вводе информации.\\
\end{enumerate}

Сценарий для прецедента «Работа с пользователями»

Основной исполнитель: Администратор.

Требования: Администратору необходимо изменить информацию о пользователе.

Предусловие: В списке есть хотя бы один пользователь.

Постусловие: Администратор изменил информацию о пользователе.

Основной успешный сценарий:
\begin{enumerate}
	\item Администратор отправил запрос на сервер об изменении информации о пользователе.
	\item Система вернула обновленное содержимое списка пользователей.
	\item Администратор получил сообщение о некорректном вводе информации.\\
\end{enumerate}

Сценарий для прецедента «Работа с пользователями»

Основной исполнитель: Администратор.

Требования: Администратору необходимо удалить пользователя.

Предусловие: В списке есть хотя бы один пользователь.

Постусловие: Администратор удалил пользователя.

Основной успешный сценарий:
\begin{enumerate}
	\item Администратор отправил запрос о удалении пользователя.
	\item Система вернула обновленное содержимое списка пользователей.
	\item Администратор получил сообщение о некорректном вводе информации.\\
\end{enumerate}

Сценарий для прецедента «Работа с продуктами»

Основной исполнитель: Пользователь.

Требования: Пользователю необходимо добавить продукт.

Предусловие: Пользователь авторизован.

Постусловие: Пользователь добавил продукт.

Основной успешный сценарий:
\begin{enumerate}
	\item Администратор отправил запрос на сервер о добавлении продукта.
	\item Система вернула обновленное содержимое списка продуктов.
	\item Администратор получил сообщение о некорректном вводе информации.\\
\end{enumerate}

Сценарий для прецедента «Работа с продуктами»

Основной исполнитель: Пользователь.

Требования: Пользователю необходимо изменить информацию о продукте.

Предусловие: В списке есть хотя бы один продукт.

Постусловие: Пользователь изменил информацию о продукте.

Основной успешный сценарий:
\begin{enumerate}
	\item Пользователь отправил запрос на сервер об изменении продукта.
	\item Система вернула обновленное содержимое списка продуктов.
	\item Администратор получил сообщение о некорректном вводе информации.\\
\end{enumerate}

Сценарий для прецедента «Работа с продуктами»

Основной исполнитель: Пользователь.

Требования: Пользователю необходимо удалить продукт.

Предусловие: В списке есть хотя бы один продукт.

Постусловие: Пользователь удалил продукт.

Основной успешный сценарий:
\begin{enumerate}
	\item Пользователь отправил запрос о удалении продукта.
	\item Система вернула обновленное содержимое списка продуктов.
	\item Администратор получил сообщение о некорректном вводе информации.\\
\end{enumerate}


Сценарий для прецедента «Работа с болезнями»

Основной исполнитель: Пользователь.

Требования: Пользователю необходимо добавить болезнь.

Предусловие: Пользователь авторизован.

Постусловие: Пользователь добавил болезнь.

Основной успешный сценарий:
\begin{enumerate}
	\item Пользователь отправил запрос на сервер о добавлении болезнь.
	\item Система вернула обновленное содержимое списка болезней.
	\item Пользователь получил сообщение о некорректном вводе информации.\\
\end{enumerate}

Сценарий для прецедента «Работа с болезнями»

Основной исполнитель: Пользователь.

Требования: Пользователю необходимо изменить информацию о болезни.

Предусловие: В списке есть хотя бы один болезнь.

Постусловие: Пользователь изменил информацию о болезни.

Основной успешный сценарий:
\begin{enumerate}
	\item Пользователь отправил запрос на сервер об изменении болезни.
	\item Система вернула обновленное содержимое списка болезни.
	\item Пользователь получил сообщение о некорректном вводе информации.\\
\end{enumerate}

Сценарий для прецедента «Работа с болезнями»

Основной исполнитель: Пользователь.

Требования: Пользователю необходимо удалить болезнь.

Предусловие: В списке есть хотя бы одна болезнь.

Постусловие: Пользователь удалил болезнь.

Основной успешный сценарий:
\begin{enumerate}
	\item Пользователь отправил запрос о удалении болезни.
	\item Система вернула обновленное содержимое списка болезни.
	\item Пользователь получил сообщение о некорректном вводе информации.\\
\end{enumerate}


Сценарий для прецедента «Работа с диетами»

Основной исполнитель: Пользователь.

Требования: Пользователю необходимо добавить диету.

Предусловие: Пользователь авторизован.

Постусловие: Пользователь добавил диету.

Основной успешный сценарий:
\begin{enumerate}
	\item Пользователь отправил запрос на сервер о добавлении диеты.
	\item Система вернула обновленное содержимое списка диет.
	\item Пользователь получил сообщение о некорректном вводе информации.\\
\end{enumerate}

Сценарий для прецедента «Работа с диетами»

Основной исполнитель: Пользователь.

Требования: Пользователю необходимо изменить информацию о диете.

Предусловие: В списке есть хотя бы одна диета.

Постусловие: Пользователь изменил информацию о диете.

Основной успешный сценарий:
\begin{enumerate}
	\item Пользователь отправил запрос на сервер об изменении диеты.
	\item Система вернула обновленное содержимое списка диет.
	\item Пользователь получил сообщение о некорректном вводе информации.\\
\end{enumerate}

Сценарий для прецедента «Работа с диетами»

Основной исполнитель: Пользователь.

Требования: Пользователю необходимо удалить диету.

Предусловие: В списке есть хотя бы одна диета.

Постусловие: Пользователь удалил диету.

Основной успешный сценарий:
\begin{enumerate}
	\item Пользователь отправил запрос о удалении диеты.
	\item Система вернула обновленное содержимое списка диет.
	\item Пользователь получил сообщение о некорректном вводе информации.\\
\end{enumerate}


Сценарий для прецедента «Работа с пациентами»

Основной исполнитель: Пользователь.

Требования: Пользователю необходимо добавить пациента.

Предусловие: Пользователь авторизован.

Постусловие: Пользователь добавил пациента.

Основной успешный сценарий:
\begin{enumerate}
	\item Пользователь отправил запрос на сервер о добавлении пациента.
	\item Система вернула обновленное содержимое списка пациентов.
	\item Пользователь получил сообщение о некорректном вводе информации.\\
\end{enumerate}

Сценарий для прецедента «Работа с пациентами»

Основной исполнитель: Пользователь.

Требования: Пользователю необходимо изменить информацию о пациенте.

Предусловие: В списке есть хотя бы одна пациент.

Постусловие: Пользователь изменил информацию о пациенте.

Основной успешный сценарий:
\begin{enumerate}
	\item Пользователь отправил запрос на сервер об изменении пациента.
	\item Система вернула обновленное содержимое списка пациентов.
	\item Пользователь получил сообщение о некорректном вводе информации.\\
\end{enumerate} 

Сценарий для прецедента «Работа с пациентами»

Основной исполнитель: Пользователь.

Требования: Пользователю необходимо удалить пациента.

Предусловие: В списке есть хотя бы один пациент.

Постусловие: Пользователь удалил пациента.

Основной успешный сценарий:
\begin{enumerate}
	\item Пользователь отправил запрос о удалении пациента.
	\item Система вернула обновленное содержимое списка пациентов.
	\item Пользователь получил сообщение о некорректном вводе информации.
\end{enumerate}

\subsection{Моделирование последовательности действий}

Сценарии для прецедентов из пункта 3.2 технического проекта можно представить как последовательность выполнения приложением определенных системных операций. Диаграммы последовательности системных операций представлены на рисунках \ref{fig:precedentAvt} -- \ref{fig:diagrammPatient} 

\begin{figure}[H]
	\centering
	\includegraphics[width=0.7\linewidth]{"images/Диаграмма последовательности системных операций для прецедента «Авторизация».drawio"}
	\caption{Диаграмма последовательности системных операций для прецедента "Авторизация"}
	\label{fig:precedentAvt}
\end{figure}

\begin{figure}[H]
	\centering
	\includegraphics[width=0.7\linewidth]{"images/Диаграмма последовательности системных операций для прецедента «Выход».drawio"}
	\caption{Диаграмма последовательности системных операций для прецедента «Выход»}
	\label{fig:diagrammaExit}
\end{figure}

\begin{figure}[H]
	\centering
	\includegraphics[width=0.7\linewidth]{"images/Диаграмма последовательности системных операций для прецедента «Диеты».drawio"}
	\caption{Диаграмма последовательности системных операций для прецедента «Диеты»}
	\label{fig:diagrammDiet}
\end{figure}

\begin{figure}[H]
	\centering
	\includegraphics[width=0.7\linewidth]{"images/Диаграмма последовательности системных операций для прецедента «Продукты».drawio"}
	\caption{Диаграмма последовательности системных операций для прецедента «Продукты»}
	\label{fig:diagrammProduct}
\end{figure}

\begin{figure}[H]
	\centering
	\includegraphics[width=0.7\linewidth]{"images/Диаграмма последовательности системных операций для прецедента Болезни.drawio"}
	\caption{Диаграмма последовательности системных операций для прецедента Болезни}
	\label{fig:diagrammDiseases}
\end{figure}

\begin{figure}[H]
	\centering
	\includegraphics[width=0.7\linewidth]{"images/Диаграмма последовательности системных операций для прецедента Пациенты.drawio"}
	\caption{Диаграмма последовательности системных операций для прецедента Пациенты}
	\label{fig:diagrammPatient}
\end{figure}

\subsection{Проектирование архитектуры программной системы}

Проанализируем прецеденты использования из пункта 3.2, а также диаграммы последовательности системных операций для прецедентов из пункта 3.3 технического проекта и определим необходимые архитектурные решения для реализации.

Разрабатываемая информационно-вычислительная система должна состоять из следующих частей:

\begin{itemize} 
	\item клиентское веб-приложение;
	\item серверное приложение;
	\item веб-сервер;
	\item база данных.
\end{itemize}

На рисунке \ref{fig:diagrammaarhitect} представлена диаграмма компонентов для разрабатываемой информационно-вычислительной системы.

\begin{figure}[H]
	\centering
	\includegraphics[width=0.7\linewidth]{"images/Диаграмма компонентов для разрабатываемой информационно-вычислительной системы.drawio"}
	\caption{Диаграмма компонентов для разрабатываемой информационно-вычислительной системы}
	\label{fig:diagrammaarhitect}
\end{figure}

Веб-приложение должно использовать компонентно-ориентированная парадигму и flux-архитектуру

Компонентый подход характеризуется делением приложения на небольшие логические части. Несколько компонентов могут использоваться в одном родительском. Таким образом строится древовидная структура
приложения. Любой компонент должен быть вызван в сценарии страницы web-сайта. Web-страница передает данные компоненту в момент вызова последнего. Компонентый подход характеризуется делением приложения на небольшие логические части.
Веб-приложение должно состоять из следующих компонентов:
\begin{itemize}
	\item App: корневой компонент;
	\item Header: компонент панели навигации;
	\item Layout:  Layout: компонент контейнер для содержимого страницы;
	\item ContentByRoute: компонент содержимого стартовой страницы;
	\item LoginForm: компонент содержимого страницы авторизации;
	\item UserCrud: компонент содержимого страницы авторизованного пользователя;
	\item ProductCrud: компонент содержимого страницы продуктов;
	\item DiseaseCrud: компонент содержимого страницы хронических болезней ЖКТ;
	\item PatientCrud: компонент содержимого страницы пациентов;
	\item DietCrud: компонент содержимого страницы диет.
\end{itemize}

На рисунке \ref{fig:comp} представлена диаграмма компонентов веб-приложения разрабатываемой информационно-вычислительной системы

\begin{figure}[H]
	\centering
	\includegraphics[width=0.7\linewidth]{"images/Диаграмма компонентов.drawio (1)"}
	\caption{Диаграмма компонентов веб-приложения}
	\label{fig:comp}
\end{figure}

В первую очередь Flux работает с информационной архитектурой, которая затем отражается в архитектуре программного обеспечения, поэтому уровень представлений обособлен и может быть легко заменен

Серверное приложение построено на событийно-ориентированной парадигме.

Серверное приложение состоит из следующих архитектурных слоев:

\begin{itemize} 
	\item сетевой дескриптор;
	\item распределяющий слой;
	\item контроллеры.
\end{itemize}

Сетевой дескриптор запускается со стартом серверного приложения и принимает соединения от клиентов по протоколу http.

Распределяющий слой определяет какой именно контроллер должен обработать тот или иной запрос в зависимости от его параметров.

Контроллеры – это конечная точка, которая занимается обработкой запроса, инициализацией новых задач, а также формированием ответа для клиента. Функции, которые содержатся в контроллерах, являются асинхронными и не блокируют обработку других запросов от клиентов.

На рисунке \ref{fig:server} представлено схематичное изображение данной архитектуры в виде диаграммы компонентов.

\begin{figure}[H]
	\centering
	\includegraphics[width=0.7\linewidth]{"images/Архитектура серверного приложения.drawio"}
	\caption{Архитектура серверного приложения}
	\label{fig:server}
\end{figure}

При вызове компонента App.tsx в сценарии web-страницы открывается окно авторизации и после успешного входа загружается главная страница Layout.tsx .

В сценарии файла UserCrud.tsx происходит вызов одного из шаблонов компонента из которого можно вызывать последующие компоненты: ProductCrud.tsx переходит на страницу продуктов, где можно добавлять новые или изменять старые, а так же удалять ненужные, siseaseCrud.tsx открывает страницу с хроническими болезнями ЖКТ, они стабильные изменить их невозможно, PatientCrud.tsx перходит на страницу пациента (карточка пациента), где хранится вся информация о больном, DietCrud.tsx открывет страницу с диетами, которые были составлены врачем-диетологом. Id шаблона также определяется в сценарии страницы web-приложения и неявно для разработчика передается. Подключается сценарий файла. 

Работа компонента заканчивается в момёент завершения работы сценария файла App.tsx, т.е. возможно выполнить действия уже после подключения шаблона.